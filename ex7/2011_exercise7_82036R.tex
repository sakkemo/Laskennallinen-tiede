% vim:tw=0
% vim:fdm=syntax
\documentclass[a4paper,12pt]{scrartcl}
\usepackage{mythesis}
\usepackage{booktabs}
%
\author{Sakari Cajanus}
\studentnumber{82036R}
\email{sakari.cajanus@aalto.fi}

\title{Exercise Round 7}{Och samma på English}
\place{Espoo}
\thesisdegree{S-114.1100 Computational Science}{Bachelor's Thesis}
%\instructor{FT Mari Myllymäki}{Ph.D. Mari Myllymäki}
%\supervisor{TkT Markus Turunen}{D.Sc. (Tech.) Markus Turunen}

\uni{Aalto-yliopisto}{Aalto University}
\school{sähkötekniikan korkeakoulu}{School of Electrical Engineering}
\degreeprogram{Bioinformaatioteknologia}{Bioinformation technology}
\department{Lääketieteellisen tekniikan ja laskennallisen tieteen laitos}{Department of Biomedical Engineering and Excellence in Computational Science}
\keywords{suomeksi}{englanniksi}

\hypersetup{%
    colorlinks=true, linktocpage=false, pdfstartpage=1, pdfstartview=FitV,%
    % uncomment the following line if you want to have black links (e.g., for printing)
    %colorlinks=false, linktocpage=false, pdfborder={0 0 0}, pdfstartpage=3, pdfstartview=FitV,% 
    breaklinks=true, pdfpagemode=UseNone, pageanchor=true, pdfpagemode=UseOutlines,%
    plainpages=false, bookmarksnumbered, bookmarksopen=true, bookmarksopenlevel=1,%
    hypertexnames=true, pdfhighlight=/O,%hyperfootnotes=true,%nesting=true,%frenchlinks,%
    urlcolor=blue, linkcolor=blue, citecolor=green, %pagecolor=RoyalBlue,%
    %urlcolor=Black, linkcolor=Black, citecolor=Black, %pagecolor=Black,%
    pdftitle={\thetitle},%the title
    pdfauthor={\theauthor},%your name
    pdfsubject={\thetitle},%
    pdfkeywords={\thekeywords},%
    pdfcreator={XeLaTeX},%
    pdfproducer={A happy XeLaTeX user}%
}

\begin{document}
%\pagenumbering{roman}
\maketitlepage
\clearpage
\pagenumbering{arabic}
\addtocounter{section}{2}
\section{Plots and conclusions}
In problem 3, we generated Gaussian random numbers using Box-Muller algorithm. We first generated $100 000$ uniformly distributed random numbers using the random number generator in python libraries, which uses the Mersenne Twister algorithm. After this, the values were transformed to follow the Gaussian distribution using
\begin{align*}
    y_{i} &= \sqrt{-2\ln x_{i}} \cos(2\pi x_{i+1})\\
    y_{i+1} &= \sqrt{-2\ln x_{i}} \sin(2\pi x_{i+1}).
\end{align*}
The results and the real shape of Gaussian distribution are shown in figure \ref{fig:gaussian}. The shape of the generated values is clearly Gaussian, and the variation is due to the values being drawn at ``random''.
\begin{figure}[h!]
  \centering
    \includegraphics[width=0.7\textwidth]{gaussian}
  \caption{Gaussian distribution generated using 10000 pseudorandom values and Box-Muller algorithm and the real shape of Gaussian distribution.}
 \label{fig:gaussian}
\end{figure}
\clearpage
\section{Plots and conclusions}
In part (a), we evaluated two pseudo random number generator algorithms by calculating moments 1-3 using different sample size $n$. The results are shown in table~\ref{tab:table}. It would seem that the LCG gets better (closer to analytically calculated) values for small $n$, but the difference gets smaller as $n$ get bigger.

\begin{table}[h!]
\caption{Results for the first three moments using different n two different PRNGs, LCG and Mersenne Twister.}
\label{tab:table}
\begin{center}
\begin{tabular}{lrrrrrrrrr}
\toprule
&& \multicolumn{2}{c}{$n=10$} & \multicolumn{2}{c}{$n=1000$} & \multicolumn{2}{c}{$n=10^5$} & \multicolumn{2}{c}{$n=10^7$} \\
\cmidrule(r){3-4} \cmidrule(r){5-6} \cmidrule(r){7-8} \cmidrule(r){9-10}
&unif & LCG & SRGL & LCG & SRGL & LCG & SRGL & LCG & SRGL\\
$\langle x\rangle$ & 0.5   & 0.5012 & 0.4585 &0.5006 &0.5172 &0.5000 & 0.5015& 0.5000 & 0.5001\\
$\langle x^2\rangle$ & 0.3333& 0.3340 & 0.2785 &0.3337 &0.3509 &0.3330 & 0.3349& 0.3330 & 0.3334\\
$\langle x^3\rangle$ & 0.25  & 0.2643 & 0.1911 &0.2504 &0.2661 &0.2495 & 0.2514& 0.2495 & 0.2501\\
\bottomrule
\end{tabular}
\end{center}
\end{table}

In part (b), we calculated 10000 values for $\langle x^2 \rangle$, and checked if the results were normally distributed as they should be according to the central limit theorem. Results are shown in figure~\ref{fig:gaussian2}, with some scaled gaussian distribution with mean of $1/3$ shown for easy comparison.
\begin{figure}[h!]
  \centering
    \includegraphics[width=0.7\textwidth]{gaussian2}
  \caption{}
 \label{fig:gaussian2}
\end{figure}

\appendix
\lstset{basicstyle=\ttfamily, numbers=left, numberstyle=\tiny, stepnumber=1, numbersep=5pt}
\gdef\thesection{Appendix \arabic{section}.}
\clearpage

\section{Code\label{LiiteA}}
\lstinputlisting[language=python]{ex5_pr4.py}
\clearpage
\section{Code\label{LiiteB}}
\lstinputlisting[language=python]{ex7_pr4.py}
\clearpage
\lstinputlisting[language=python]{ex7_pr4_b.py}
\end{document}
