% vim:tw=0
% vim:fdm=syntax
\documentclass[a4paper,12pt]{scrartcl}
\usepackage{mythesis}
%
\author{Sakari Cajanus}
\studentnumber{82036R}
\email{sakari.cajanus@aalto.fi}

\title{Exercise Round 4}{Och samma på English}
\place{Espoo}
\thesisdegree{S-114.1100 Computational Science}{Bachelor's Thesis}
%\instructor{FT Mari Myllymäki}{Ph.D. Mari Myllymäki}
%\supervisor{TkT Markus Turunen}{D.Sc. (Tech.) Markus Turunen}

\uni{Aalto-yliopisto}{Aalto University}
\school{sähkötekniikan korkeakoulu}{School of Electrical Engineering}
\degreeprogram{Bioinformaatioteknologia}{Bioinformation technology}
\department{Lääketieteellisen tekniikan ja laskennallisen tieteen laitos}{Department of Biomedical Engineering and Excellence in Computational Science}
\keywords{suomeksi}{englanniksi}

\hypersetup{%
    colorlinks=true, linktocpage=false, pdfstartpage=1, pdfstartview=FitV,%
    % uncomment the following line if you want to have black links (e.g., for printing)
    %colorlinks=false, linktocpage=false, pdfborder={0 0 0}, pdfstartpage=3, pdfstartview=FitV,% 
    breaklinks=true, pdfpagemode=UseNone, pageanchor=true, pdfpagemode=UseOutlines,%
    plainpages=false, bookmarksnumbered, bookmarksopen=true, bookmarksopenlevel=1,%
    hypertexnames=true, pdfhighlight=/O,%hyperfootnotes=true,%nesting=true,%frenchlinks,%
    urlcolor=blue, linkcolor=blue, citecolor=green, %pagecolor=RoyalBlue,%
    %urlcolor=Black, linkcolor=Black, citecolor=Black, %pagecolor=Black,%
    pdftitle={\thetitle},%the title
    pdfauthor={\theauthor},%your name
    pdfsubject={\thetitle},%
    pdfkeywords={\thekeywords},%
    pdfcreator={XeLaTeX},%
    pdfproducer={A happy XeLaTeX user}%
}

\begin{document}
%\pagenumbering{roman}
\maketitlepage
\clearpage
\pagenumbering{arabic}
\section{Plots and conclusions}
In problem 3, a method for solving systems of linear equations using Gaussian elimination with partial pivoting was programmed. The code is given in appendix \ref{LiiteA}. This was in turn used to solve a group of linear problems, each consisting of Hilbert matrix $\mathbf{A}$ defined by
\begin{align*}
    a_{ij}=(i+j-1)^{-1}\hspace{5mm}\textrm{for } 1\le i,j \le n
\end{align*}
and vector $\mathbf{b}$, $b_i=\sum_{j=1}^n a_{ij}x_j$. in the following printout are the solution vectors $\mathbf{x}$ of equation
\begin{align*}
    \mathbf{Ax}=\mathbf{b}.
\end{align*}
Scaling partial pivoting was used: This can be enabled/disabled in the code.
\begin{verbatim}
[[ 1.  1.]]
[[ 1.  1.  1.]]
[[ 1.  1.  1.  1.]]
[[ 1.  1.  1.  1.  1.]]
[[ 1.  1.  1.  1.  1.  1.]]
[[ 1.          1.          1.          1.00000001  0.99999998  1.00000001
   1.        ]]

[[ 1.          1.          0.99999997  1.00000015  0.9999996   1.00000057
   0.9999996   1.00000011]]

[[ 1.          1.00000002  0.99999966  1.00000244  0.99999106  1.00001821
   0.99997913  1.00001258  0.99999689]]

[[ 1.          1.00000005  0.99999885  1.00001037  0.99995072  1.00013499
   0.99977918  1.00021284  0.99988852  1.00002447]]

[[ 0.99999999  1.00000101  0.99997361  1.00029594  0.99823198  1.00623157
   0.98640193  1.01857428  0.98454439  1.00716206  0.99858323]]

[[ 0.99999998  1.00000195  0.9999404   1.00079437  0.99428063  1.02475441
   0.93190112  1.12192751  0.85840445  1.10284426  0.95755267  1.00759824]]

[[  1.00000015   0.99997701   1.00089269   0.98505106   1.13488627
    0.26533343   3.57220075  -4.98296516  10.342891    -8.68096223
    7.38324107  -1.42411822   1.40357233]]

[[ 1.00000001  0.99999732  1.00012225  0.99764671  1.02424519  0.84897153
   1.60987249 -0.661489    4.11355401 -3.02089848  4.51628292 -0.98866436
   1.65637131  0.9039881 ]]

[[ 0.99999999  0.99999883  1.00012965  0.99647227  1.0440312   0.69300071
   2.30907586 -2.54662614  7.06382489 -4.98285744  2.95005877  3.64031085
  -2.66944063  2.86278794  0.63923316]]

\end{verbatim}
As the size of the matrix $\mathbf{A}$ increases, the error in calculating the matrix $\mathbf{b}$ is amplified. This happens because as the size of the matrix increases, the smallest value in the matrix decreases: For the matrix of size two, the smallest value in the Hilbert matrix is $\frac{1}{3}$, for a matrix of size $15$, the smallest value is $\frac{1}{29}$.

The problem with the calculation can be shown, for example, when using only 3 significant digits, while $\mathbf{A}$ is a $3\times3$ Hilbert matrix. The values are rounded only when they're written as below. (When calculating new rows, exact values coming from the division by pivot are used.)
\begin{align*}
    \begin{bmatrix}
        1.000 & 0.500 & 0.333 \\
        0.500 & 0.333 & 0.250 \\
        0.333 & 0.250 & 0.200 \\
    \end{bmatrix}
    \begin{bmatrix}
        x_1 \\
        x_2 \\
        x_3 \\
    \end{bmatrix}
=
    \begin{bmatrix}
        1.83 \\
        1.08 \\
        0.783 \\
    \end{bmatrix}
\end{align*}
\begin{align*}
    \begin{bmatrix}
        1.000 & 0.500 & 0.333 \\
        0.000 & 0.0830 & 0.0835 \\
        0.000 & 0.0835 & 0.0891 \\
    \end{bmatrix}
    \begin{bmatrix}
        x_1 \\
        x_2 \\
        x_3 \\
    \end{bmatrix}
=
    \begin{bmatrix}
        1.83 \\
        0.165 \\
        0.174 \\
    \end{bmatrix}
\end{align*}
\begin{align*}
    \begin{bmatrix}
        1.000 & 0.500 & 0.333 \\
        0.000 & 0.0830 & 0.0835 \\
        0.000 & 0.000 & 0.00511 \\
    \end{bmatrix}
    \begin{bmatrix}
        x_1 \\
        x_2 \\
        x_3 \\
    \end{bmatrix}
=
    \begin{bmatrix}
        1.83 \\
        0.165 \\
        0.00762 \\
    \end{bmatrix}
\end{align*}
\begin{align*}
    \begin{bmatrix}
        x_1 \\
        x_2 \\
        x_3 \\
    \end{bmatrix}
=
    \begin{bmatrix}
        1.09 \\
        0.489 \\
        1.49 \\
    \end{bmatrix}
\end{align*}
The problems in the calculations arise from substracting (nearly equal?) values.
\clearpage
\appendix
\lstset{basicstyle=\ttfamily, numbers=left, numberstyle=\tiny, stepnumber=1, numbersep=5pt}
\gdef\thesection{Appendix \arabic{section}.}
\clearpage

\section{Code\label{LiiteA}}
\lstinputlisting[language=python]{ex4_pr3.py}
\clearpage
\end{document}
