% vim:tw=0
% vim:fdm=syntax
\documentclass[a4paper,12pt]{scrartcl}
\usepackage{mythesis}
%
\author{Sakari Cajanus}
\studentnumber{82036R}
\email{sakari.cajanus@aalto.fi}

\title{Exercise Round 5}{Och samma på English}
\place{Espoo}
\thesisdegree{S-114.1100 Computational Science}{Bachelor's Thesis}
%\instructor{FT Mari Myllymäki}{Ph.D. Mari Myllymäki}
%\supervisor{TkT Markus Turunen}{D.Sc. (Tech.) Markus Turunen}

\uni{Aalto-yliopisto}{Aalto University}
\school{sähkötekniikan korkeakoulu}{School of Electrical Engineering}
\degreeprogram{Bioinformaatioteknologia}{Bioinformation technology}
\department{Lääketieteellisen tekniikan ja laskennallisen tieteen laitos}{Department of Biomedical Engineering and Excellence in Computational Science}
\keywords{suomeksi}{englanniksi}

\hypersetup{%
    colorlinks=true, linktocpage=false, pdfstartpage=1, pdfstartview=FitV,%
    % uncomment the following line if you want to have black links (e.g., for printing)
    %colorlinks=false, linktocpage=false, pdfborder={0 0 0}, pdfstartpage=3, pdfstartview=FitV,% 
    breaklinks=true, pdfpagemode=UseNone, pageanchor=true, pdfpagemode=UseOutlines,%
    plainpages=false, bookmarksnumbered, bookmarksopen=true, bookmarksopenlevel=1,%
    hypertexnames=true, pdfhighlight=/O,%hyperfootnotes=true,%nesting=true,%frenchlinks,%
    urlcolor=blue, linkcolor=blue, citecolor=green, %pagecolor=RoyalBlue,%
    %urlcolor=Black, linkcolor=Black, citecolor=Black, %pagecolor=Black,%
    pdftitle={\thetitle},%the title
    pdfauthor={\theauthor},%your name
    pdfsubject={\thetitle},%
    pdfkeywords={\thekeywords},%
    pdfcreator={XeLaTeX},%
    pdfproducer={A happy XeLaTeX user}%
}

\begin{document}
%\pagenumbering{roman}
\maketitlepage
\clearpage
\pagenumbering{arabic}
\addtocounter{section}{4}
\section{Plots and conclusions}
Figures show the calculated values, code is given in the appendices. The matlab calculation was made using fuction \emph{csapi}: because of this, there's slight difference in endpoints. The matlab toolbox Spline built-in function csapi uses not-a-knot end conditions while the python code assumes that the function continues linearly (the second derivative is zero).

In the not-a-knot end condition case, third derivative between last and second last points is assumed to be continuous (and of course constant, in the case of cubic spline).
\begin{figure}[h!]
  \centering
    \includegraphics[width=0.7\textwidth]{function}
  \caption{Estimated function values from the python code}
  \label{fig:function}
\end{figure}
\begin{figure}[h!]
  \centering
    \includegraphics[width=0.7\textwidth]{derivative}
  \caption{Estimated derivative values from the python code}
  \label{fig:derivative}
\end{figure}
\clearpage
\begin{figure}[h!]
  \centering
    \includegraphics[width=0.7\textwidth]{function_m}
  \caption{Estimated function values from Matlab}
  \label{fig:function_m}
\end{figure}
\begin{figure}[h!]
  \centering
    \includegraphics[width=0.7\textwidth]{derivative_m}
  \caption{Estimated derivative values from Matlab}
  \label{fig:derivative_m}
\end{figure}
\clearpage
\appendix
\lstset{basicstyle=\ttfamily, numbers=left, numberstyle=\tiny, stepnumber=1, numbersep=5pt}
\gdef\thesection{Appendix \arabic{section}.}
\clearpage

\section{Code\label{LiiteA}}
\lstinputlisting[language=python]{ex5_pr4.py}
\clearpage
\section{Code\label{LiiteB}}
\lstinputlisting[language=matlab]{spline.m}
\clearpage
\end{document}
