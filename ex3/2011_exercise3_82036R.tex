% vim:tw=0
% vim:fdm=syntax
\documentclass[a4paper,12pt]{scrartcl}
\usepackage{mythesis}
%
\author{Sakari Cajanus}
\studentnumber{82036R}
\email{sakari.cajanus@aalto.fi}

\title{Exercise Round 2}{Och samma på English}
\place{Espoo}
\thesisdegree{S-114.1100 Computational Science}{Bachelor's Thesis}
%\instructor{FT Mari Myllymäki}{Ph.D. Mari Myllymäki}
%\supervisor{TkT Markus Turunen}{D.Sc. (Tech.) Markus Turunen}

\uni{Aalto-yliopisto}{Aalto University}
\school{sähkötekniikan korkeakoulu}{School of Electrical Engineering}
\degreeprogram{Bioinformaatioteknologia}{Bioinformation technology}
\department{Lääketieteellisen tekniikan ja laskennallisen tieteen laitos}{Department of Biomedical Engineering and Excellence in Computational Science}
\keywords{suomeksi}{englanniksi}

\hypersetup{%
    colorlinks=true, linktocpage=false, pdfstartpage=1, pdfstartview=FitV,%
    % uncomment the following line if you want to have black links (e.g., for printing)
    %colorlinks=false, linktocpage=false, pdfborder={0 0 0}, pdfstartpage=3, pdfstartview=FitV,% 
    breaklinks=true, pdfpagemode=UseNone, pageanchor=true, pdfpagemode=UseOutlines,%
    plainpages=false, bookmarksnumbered, bookmarksopen=true, bookmarksopenlevel=1,%
    hypertexnames=true, pdfhighlight=/O,%hyperfootnotes=true,%nesting=true,%frenchlinks,%
    urlcolor=blue, linkcolor=blue, citecolor=green, %pagecolor=RoyalBlue,%
    %urlcolor=Black, linkcolor=Black, citecolor=Black, %pagecolor=Black,%
    pdftitle={\thetitle},%the title
    pdfauthor={\theauthor},%your name
    pdfsubject={\thetitle},%
    pdfkeywords={\thekeywords},%
    pdfcreator={XeLaTeX},%
    pdfproducer={A happy XeLaTeX user}%
}

\begin{document}
%\pagenumbering{roman}
\maketitlepage
\clearpage
\pagenumbering{arabic}
\section{Plots and conclusions}
In problem 3 we were asked to evaluate three integrals
\begin{enumerate}[(i.)]
    \item $\displaystyle \int_0^2 \! \frac{1}{1+x} \, \mathrm{d} x = \ln{(1+x)}\bigg\rvert_{x=0}^2$\\
    \item $\displaystyle \int_0^1 \! \mathrm{e}^x \, \mathrm{d} x = \mathrm{e}^x\bigg\rvert_{x=0}^1$\\
    \item $\displaystyle \int_0^1 \! \sqrt{x} \, \mathrm{d} x = \frac{2}{3}x^{3/2}\bigg\rvert_{x=0}^1$
\end{enumerate}
using Romberg algorithm. The exact value of the integrals was known: the
value calculated using Romberg algorithm was then compared to the exact
value.

For the first two integrals the approximation R(5,5) is good:
\begin{verbatim}
1.33333333 0.00000000 0.00000000 0.00000000 0.00000000 0.0000000000
1.16666667 1.11111111 0.00000000 0.00000000 0.00000000 0.0000000000
1.11666667 1.10000000 1.09925926 0.00000000 0.00000000 0.0000000000
1.10321068 1.09872535 1.09864037 1.09863055 0.00000000 0.0000000000
1.09976770 1.09862004 1.09861302 1.09861259 1.09861252 0.0000000000
1.09890152 1.09861279 1.09861230 1.09861229 1.09861229 1.0986122898
True value: 1.0986122887
\end{verbatim}
The first integral is correct to 8 decimals.
\begin{verbatim}
1.85914091 0.00000000 0.00000000 0.00000000 0.00000000 0.00000000000000
1.75393109 1.71886115 0.00000000 0.00000000 0.00000000 0.00000000000000
1.72722190 1.71831884 1.71828269 0.00000000 0.00000000 0.00000000000000
1.72051859 1.71828415 1.71828184 1.71828183 0.00000000 0.00000000000000
1.71884113 1.71828197 1.71828183 1.71828183 1.71828183 0.00000000000000
1.71842166 1.71828184 1.71828183 1.71828183 1.71828183 1.71828182845905
True value: 1.71828182845905
\end{verbatim}
The second integral is correct to (at least) 14 decimals. However,
\begin{verbatim}
0.500000 0.000000 0.000000 0.000000 0.000000 0.000000 0.000000 0.000000 0.00000000
0.603553 0.638071 0.000000 0.000000 0.000000 0.000000 0.000000 0.000000 0.00000000
0.643283 0.656526 0.657757 0.000000 0.000000 0.000000 0.000000 0.000000 0.00000000
0.658130 0.663079 0.663516 0.663608 0.000000 0.000000 0.000000 0.000000 0.00000000
0.663581 0.665398 0.665553 0.665585 0.665593 0.000000 0.000000 0.000000 0.00000000
0.665559 0.666218 0.666273 0.666284 0.666287 0.666288 0.000000 0.000000 0.00000000
0.666271 0.666508 0.666527 0.666531 0.666532 0.666533 0.666533 0.000000 0.00000000
0.666526 0.666611 0.666617 0.666619 0.666619 0.666619 0.666619 0.666619 0.00000000
0.666617 0.666647 0.666649 0.666650 0.666650 0.666650 0.666650 0.666650 0.66664993
True value: 0.6666666667

\end{verbatim}
the third integral is only correct to 4 decimals, even when using
Romberg array of size R(8,8). This is because the derivative of the
function $\sqrt{x}$ 
\begin{align*}
    \frac{\mathrm{d}}{\mathrm{d}x}\sqrt{x}=\frac{1}{2\sqrt{x}}
\end{align*}
cannot be calculated at $x=0$. Figure \ref{fig:derivative}
shows the behavious of the derivative in our range $[0,1]$
\begin{figure}[h!]
  \centering
    \includegraphics[width=0.7\textwidth]{derivative}
  \caption{As $x$ approaches zero from right, the derivative approaches
infinity.}
  \label{fig:derivative}
\end{figure}

\noindent Evaluating the integral from 0.1 (where the derivative already
gets ``sane'' values) to 1, the problems dissipate:
\begin{verbatim}
0.592302 0.000000 0.000000 0.000000 0.000000 0.000000 0.000000 0.000000 0.0000000000
0.629880 0.642406 0.000000 0.000000 0.000000 0.000000 0.000000 0.000000 0.0000000000
0.641287 0.645089 0.645267 0.000000 0.000000 0.000000 0.000000 0.000000 0.0000000000
0.644466 0.645526 0.645556 0.645560 0.000000 0.000000 0.000000 0.000000 0.0000000000
0.645301 0.645580 0.645583 0.645584 0.645584 0.000000 0.000000 0.000000 0.0000000000
0.645514 0.645584 0.645585 0.645585 0.645585 0.645585 0.000000 0.000000 0.0000000000
0.645567 0.645585 0.645585 0.645585 0.645585 0.645585 0.645585 0.000000 0.0000000000
0.645580 0.645585 0.645585 0.645585 0.645585 0.645585 0.645585 0.645585 0.0000000000
0.645584 0.645585 0.645585 0.645585 0.645585 0.645585 0.645585 0.645585 0.6455848156
True value: 0.6455848156
\end{verbatim}
\clearpage
\section{Plots and Conclusions}
Figure \ref{fig:partition1} shows the behaviour of Simpson algorithm for
evaluating integral
\begin{align}\label{eka}
    \int_0^1\frac{4}{1+x^2}\mathrm{d}x.
\end{align}
The divisions shown show the Simpson's rule division of one Simpson's rule,
to which the function value calculated using two Simpson's rules is
compared. This means that the evaluations are one step size more accurate
than the figures shown.

Similarly, figure \ref{fig:partition2} shows the same procedure for integral
\begin{align}\label{toka}
    \int_0^{2\pi}\frac{\cos(2x)}{\mathrm{e}^x}\mathrm{d}x.
\end{align}
\begin{figure}[h!]
  \centering
    \includegraphics[width=.9\textwidth]{partition1}
\caption{Behaviour of adaptive Simpson's algorithm for integral \ref{eka}.}
  \label{fig:partition1}
\end{figure}
\clearpage
\begin{figure}[h!]
  \centering
    \includegraphics[width=.9\textwidth]{partition2}
\caption{Behaviour of adaptive Simpson's algorithm for integral \ref{toka}.}
  \label{fig:partition2}
\end{figure}
\clearpage
\appendix
\lstset{basicstyle=\ttfamily, numbers=left, numberstyle=\tiny, stepnumber=1, numbersep=5pt}
\gdef\thesection{Appendix \arabic{section}.}
\clearpage

\section{Code\label{LiiteA}}
\lstinputlisting[language=python]{ex3_pr3.py}
\clearpage
\section{Code\label{LiiteB}}
\lstinputlisting[language=python]{ex3_pr5.py}
\end{document}
